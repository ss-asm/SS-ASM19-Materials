\documentclass[]{article}
\usepackage{lmodern}
\usepackage{amssymb,amsmath}
\usepackage{ifxetex,ifluatex}
\usepackage{fixltx2e} % provides \textsubscript
\ifnum 0\ifxetex 1\fi\ifluatex 1\fi=0 % if pdftex
  \usepackage[T1]{fontenc}
  \usepackage[utf8]{inputenc}
\else % if luatex or xelatex
  \ifxetex
    \usepackage{mathspec}
  \else
    \usepackage{fontspec}
  \fi
  \defaultfontfeatures{Ligatures=TeX,Scale=MatchLowercase}
\fi
% use upquote if available, for straight quotes in verbatim environments
\IfFileExists{upquote.sty}{\usepackage{upquote}}{}
% use microtype if available
\IfFileExists{microtype.sty}{%
\usepackage{microtype}
\UseMicrotypeSet[protrusion]{basicmath} % disable protrusion for tt fonts
}{}
\usepackage[margin=1in]{geometry}
\usepackage{hyperref}
\hypersetup{unicode=true,
            pdftitle={Remote Sensing Tutorial 1-3},
            pdfauthor={Matthew Ng, Maxim Chernetskiy, Andrew MacLachlan},
            pdfborder={0 0 0},
            breaklinks=true}
\urlstyle{same}  % don't use monospace font for urls
\usepackage{color}
\usepackage{fancyvrb}
\newcommand{\VerbBar}{|}
\newcommand{\VERB}{\Verb[commandchars=\\\{\}]}
\DefineVerbatimEnvironment{Highlighting}{Verbatim}{commandchars=\\\{\}}
% Add ',fontsize=\small' for more characters per line
\usepackage{framed}
\definecolor{shadecolor}{RGB}{248,248,248}
\newenvironment{Shaded}{\begin{snugshade}}{\end{snugshade}}
\newcommand{\AlertTok}[1]{\textcolor[rgb]{0.94,0.16,0.16}{#1}}
\newcommand{\AnnotationTok}[1]{\textcolor[rgb]{0.56,0.35,0.01}{\textbf{\textit{#1}}}}
\newcommand{\AttributeTok}[1]{\textcolor[rgb]{0.77,0.63,0.00}{#1}}
\newcommand{\BaseNTok}[1]{\textcolor[rgb]{0.00,0.00,0.81}{#1}}
\newcommand{\BuiltInTok}[1]{#1}
\newcommand{\CharTok}[1]{\textcolor[rgb]{0.31,0.60,0.02}{#1}}
\newcommand{\CommentTok}[1]{\textcolor[rgb]{0.56,0.35,0.01}{\textit{#1}}}
\newcommand{\CommentVarTok}[1]{\textcolor[rgb]{0.56,0.35,0.01}{\textbf{\textit{#1}}}}
\newcommand{\ConstantTok}[1]{\textcolor[rgb]{0.00,0.00,0.00}{#1}}
\newcommand{\ControlFlowTok}[1]{\textcolor[rgb]{0.13,0.29,0.53}{\textbf{#1}}}
\newcommand{\DataTypeTok}[1]{\textcolor[rgb]{0.13,0.29,0.53}{#1}}
\newcommand{\DecValTok}[1]{\textcolor[rgb]{0.00,0.00,0.81}{#1}}
\newcommand{\DocumentationTok}[1]{\textcolor[rgb]{0.56,0.35,0.01}{\textbf{\textit{#1}}}}
\newcommand{\ErrorTok}[1]{\textcolor[rgb]{0.64,0.00,0.00}{\textbf{#1}}}
\newcommand{\ExtensionTok}[1]{#1}
\newcommand{\FloatTok}[1]{\textcolor[rgb]{0.00,0.00,0.81}{#1}}
\newcommand{\FunctionTok}[1]{\textcolor[rgb]{0.00,0.00,0.00}{#1}}
\newcommand{\ImportTok}[1]{#1}
\newcommand{\InformationTok}[1]{\textcolor[rgb]{0.56,0.35,0.01}{\textbf{\textit{#1}}}}
\newcommand{\KeywordTok}[1]{\textcolor[rgb]{0.13,0.29,0.53}{\textbf{#1}}}
\newcommand{\NormalTok}[1]{#1}
\newcommand{\OperatorTok}[1]{\textcolor[rgb]{0.81,0.36,0.00}{\textbf{#1}}}
\newcommand{\OtherTok}[1]{\textcolor[rgb]{0.56,0.35,0.01}{#1}}
\newcommand{\PreprocessorTok}[1]{\textcolor[rgb]{0.56,0.35,0.01}{\textit{#1}}}
\newcommand{\RegionMarkerTok}[1]{#1}
\newcommand{\SpecialCharTok}[1]{\textcolor[rgb]{0.00,0.00,0.00}{#1}}
\newcommand{\SpecialStringTok}[1]{\textcolor[rgb]{0.31,0.60,0.02}{#1}}
\newcommand{\StringTok}[1]{\textcolor[rgb]{0.31,0.60,0.02}{#1}}
\newcommand{\VariableTok}[1]{\textcolor[rgb]{0.00,0.00,0.00}{#1}}
\newcommand{\VerbatimStringTok}[1]{\textcolor[rgb]{0.31,0.60,0.02}{#1}}
\newcommand{\WarningTok}[1]{\textcolor[rgb]{0.56,0.35,0.01}{\textbf{\textit{#1}}}}
\usepackage{graphicx,grffile}
\makeatletter
\def\maxwidth{\ifdim\Gin@nat@width>\linewidth\linewidth\else\Gin@nat@width\fi}
\def\maxheight{\ifdim\Gin@nat@height>\textheight\textheight\else\Gin@nat@height\fi}
\makeatother
% Scale images if necessary, so that they will not overflow the page
% margins by default, and it is still possible to overwrite the defaults
% using explicit options in \includegraphics[width, height, ...]{}
\setkeys{Gin}{width=\maxwidth,height=\maxheight,keepaspectratio}
\IfFileExists{parskip.sty}{%
\usepackage{parskip}
}{% else
\setlength{\parindent}{0pt}
\setlength{\parskip}{6pt plus 2pt minus 1pt}
}
\setlength{\emergencystretch}{3em}  % prevent overfull lines
\providecommand{\tightlist}{%
  \setlength{\itemsep}{0pt}\setlength{\parskip}{0pt}}
\setcounter{secnumdepth}{0}
% Redefines (sub)paragraphs to behave more like sections
\ifx\paragraph\undefined\else
\let\oldparagraph\paragraph
\renewcommand{\paragraph}[1]{\oldparagraph{#1}\mbox{}}
\fi
\ifx\subparagraph\undefined\else
\let\oldsubparagraph\subparagraph
\renewcommand{\subparagraph}[1]{\oldsubparagraph{#1}\mbox{}}
\fi

%%% Use protect on footnotes to avoid problems with footnotes in titles
\let\rmarkdownfootnote\footnote%
\def\footnote{\protect\rmarkdownfootnote}

%%% Change title format to be more compact
\usepackage{titling}

% Create subtitle command for use in maketitle
\providecommand{\subtitle}[1]{
  \posttitle{
    \begin{center}\large#1\end{center}
    }
}

\setlength{\droptitle}{-2em}

  \title{Remote Sensing Tutorial 1-3}
    \pretitle{\vspace{\droptitle}\centering\huge}
  \posttitle{\par}
    \author{Matthew Ng, Maxim Chernetskiy, Andrew MacLachlan}
    \preauthor{\centering\large\emph}
  \postauthor{\par}
      \predate{\centering\large\emph}
  \postdate{\par}
    \date{7/22/2019}


\begin{document}
\maketitle

\hypertarget{introduction-to-raster-data-and-analysis}{%
\section{Introduction to Raster Data and
Analysis}\label{introduction-to-raster-data-and-analysis}}

\hypertarget{learning-objectives}{%
\subsection{Learning Objectives}\label{learning-objectives}}

\begin{itemize}
\tightlist
\item
  What is raster data?
\item
  Where to get raster data?
\item
  Basic information about your datasets
\item
  Mathematical applications to each dataset
\end{itemize}

In essence,

\begin{Shaded}
\begin{Highlighting}[]
\KeywordTok{setwd}\NormalTok{(}\StringTok{"C:/Users/mattk/OneDrive/Desktop/rs"}\NormalTok{)}
\KeywordTok{getwd}\NormalTok{()}
\end{Highlighting}
\end{Shaded}

\begin{verbatim}
## [1] "C:/Users/mattk/OneDrive/Desktop/rs"
\end{verbatim}

\begin{Shaded}
\begin{Highlighting}[]
\KeywordTok{list.files}\NormalTok{()}
\end{Highlighting}
\end{Shaded}

\begin{verbatim}
##  [1] "clip_LC08_L1TP_203023_20190513_20190521_01_T1_B1.tif" 
##  [2] "clip_LC08_L1TP_203023_20190513_20190521_01_T1_B10.tif"
##  [3] "clip_LC08_L1TP_203023_20190513_20190521_01_T1_B11.tif"
##  [4] "clip_LC08_L1TP_203023_20190513_20190521_01_T1_B2.tif" 
##  [5] "clip_LC08_L1TP_203023_20190513_20190521_01_T1_B3.tif" 
##  [6] "clip_LC08_L1TP_203023_20190513_20190521_01_T1_B4.tif" 
##  [7] "clip_LC08_L1TP_203023_20190513_20190521_01_T1_B5.tif" 
##  [8] "clip_LC08_L1TP_203023_20190513_20190521_01_T1_B6.tif" 
##  [9] "clip_LC08_L1TP_203023_20190513_20190521_01_T1_B7.tif" 
## [10] "clip_LC08_L1TP_203023_20190513_20190521_01_T1_B8.tif" 
## [11] "clip_LC08_L1TP_203023_20190513_20190521_01_T1_B9.tif" 
## [12] "Compiled RS Summer School Notebook.Rmd"               
## [13] "Compiled_RS_Summer_School_Notebook.Rmd"               
## [14] "LC08_L1TP_203023_20190513_20190521_01_T1_B1.TIF"      
## [15] "LC08_L1TP_203023_20190513_20190521_01_T1_B10.TIF"     
## [16] "LC08_L1TP_203023_20190513_20190521_01_T1_B11.TIF"     
## [17] "LC08_L1TP_203023_20190513_20190521_01_T1_B2.TIF"      
## [18] "LC08_L1TP_203023_20190513_20190521_01_T1_B3.TIF"      
## [19] "LC08_L1TP_203023_20190513_20190521_01_T1_B4.TIF"      
## [20] "LC08_L1TP_203023_20190513_20190521_01_T1_B5.TIF"      
## [21] "LC08_L1TP_203023_20190513_20190521_01_T1_B6.TIF"      
## [22] "LC08_L1TP_203023_20190513_20190521_01_T1_B7.TIF"      
## [23] "LC08_L1TP_203023_20190513_20190521_01_T1_B8.TIF"      
## [24] "LC08_L1TP_203023_20190513_20190521_01_T1_B9.TIF"      
## [25] "manchester_boundary.cpg"                              
## [26] "manchester_boundary.dbf"                              
## [27] "manchester_boundary.prj"                              
## [28] "manchester_boundary.sbn"                              
## [29] "manchester_boundary.sbx"                              
## [30] "manchester_boundary.shp"                              
## [31] "manchester_boundary.shp.xml"                          
## [32] "manchester_boundary.shx"                              
## [33] "RS Sample Notebook.Rmd"                               
## [34] "RS_Sample_Notebook.html"
\end{verbatim}

\hypertarget{setup-environment}{%
\section{Setup Environment}\label{setup-environment}}

These are all the required libraries required for the next three
remote-sensing tutorials.

\begin{Shaded}
\begin{Highlighting}[]
\KeywordTok{library}\NormalTok{(sp)}
\end{Highlighting}
\end{Shaded}

\begin{verbatim}
## Warning: package 'sp' was built under R version 3.6.1
\end{verbatim}

\begin{Shaded}
\begin{Highlighting}[]
\KeywordTok{library}\NormalTok{(RColorBrewer)}
\KeywordTok{library}\NormalTok{(lattice)}
\end{Highlighting}
\end{Shaded}

\begin{verbatim}
## Warning: package 'lattice' was built under R version 3.6.1
\end{verbatim}

\begin{Shaded}
\begin{Highlighting}[]
\KeywordTok{library}\NormalTok{(latticeExtra)}
\end{Highlighting}
\end{Shaded}

\begin{verbatim}
## Warning: package 'latticeExtra' was built under R version 3.6.1
\end{verbatim}

\begin{Shaded}
\begin{Highlighting}[]
\KeywordTok{library}\NormalTok{(raster)}
\end{Highlighting}
\end{Shaded}

\begin{verbatim}
## Warning: package 'raster' was built under R version 3.6.1
\end{verbatim}

\begin{Shaded}
\begin{Highlighting}[]
\KeywordTok{library}\NormalTok{(rgeos)}
\end{Highlighting}
\end{Shaded}

\begin{verbatim}
## Warning: package 'rgeos' was built under R version 3.6.1
\end{verbatim}

\begin{verbatim}
## rgeos version: 0.4-3, (SVN revision 595)
##  GEOS runtime version: 3.6.1-CAPI-1.10.1 
##  Linking to sp version: 1.3-1 
##  Polygon checking: TRUE
\end{verbatim}

\begin{Shaded}
\begin{Highlighting}[]
\KeywordTok{library}\NormalTok{(rgdal)}
\end{Highlighting}
\end{Shaded}

\begin{verbatim}
## Warning: package 'rgdal' was built under R version 3.6.1
\end{verbatim}

\begin{verbatim}
## rgdal: version: 1.4-4, (SVN revision 833)
##  Geospatial Data Abstraction Library extensions to R successfully loaded
##  Loaded GDAL runtime: GDAL 2.2.3, released 2017/11/20
##  Path to GDAL shared files: C:/Users/mattk/OneDrive/Documents/R/win-library/3.6/rgdal/gdal
##  GDAL binary built with GEOS: TRUE 
##  Loaded PROJ.4 runtime: Rel. 4.9.3, 15 August 2016, [PJ_VERSION: 493]
##  Path to PROJ.4 shared files: C:/Users/mattk/OneDrive/Documents/R/win-library/3.6/rgdal/proj
##  Linking to sp version: 1.3-1
\end{verbatim}

\begin{Shaded}
\begin{Highlighting}[]
\KeywordTok{library}\NormalTok{(rasterVis)}
\end{Highlighting}
\end{Shaded}

\begin{verbatim}
## Warning: package 'rasterVis' was built under R version 3.6.1
\end{verbatim}

\begin{Shaded}
\begin{Highlighting}[]
\KeywordTok{library}\NormalTok{(ggplot2)}
\end{Highlighting}
\end{Shaded}

\begin{verbatim}
## 
## Attaching package: 'ggplot2'
\end{verbatim}

\begin{verbatim}
## The following object is masked from 'package:latticeExtra':
## 
##     layer
\end{verbatim}


\end{document}
